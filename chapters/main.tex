\documentclass[aspectratio=169,xcolor=dvipsnames]{beamer}

% Beamer theme and customization
\usetheme{Madrid}
\usecolortheme{whale}

% Packages
\usepackage[utf8]{inputenc}
\usepackage[T1]{fontenc}
\usepackage{lmodern}
\usepackage{graphicx}
\usepackage{amsmath,amsfonts,amssymb}
\usepackage{listings}
\usepackage{xcolor}
\usepackage{tikz}
\usepackage{hyperref}
\usepackage{booktabs}
\usepackage{array}
\usepackage{multicol}
\usepackage{fancyvrb}

% TikZ libraries
\usetikzlibrary{shapes,arrows,positioning,calc,decorations.pathreplacing}

% Code listing settings
\lstset{
    language=C,
    basicstyle=\tiny\ttfamily,
    keywordstyle=\color{blue}\bfseries,
    commentstyle=\color{gray}\itshape,
    stringstyle=\color{red},
    numbers=left,
    numberstyle=\tiny\color{gray},
    stepnumber=1,
    numbersep=5pt,
    backgroundcolor=\color{lightgray!10},
    showspaces=false,
    showstringspaces=false,
    showtabs=false,
    frame=single,
    frameround=tttt,
    rulecolor=\color{black!30},
    breaklines=true,
    breakatwhitespace=false,
    tabsize=2
}

% Custom colors
\definecolor{lightblue}{RGB}{173,216,230}
\definecolor{darkblue}{RGB}{0,90,156}
\definecolor{lightgray}{RGB}{245,245,245}
\definecolor{darkgray}{RGB}{128,128,128}

% Presentation information
\title[System Design \& Computer Architecture]{System Design and Computer Architecture}
\subtitle{Understanding Modern Computing Systems}
\author[Your Name]{Your Name}
\institute[Institution]{
    Department of Computer Science\\
    Your Institution
}
\date{\today}

% Custom commands for consistency
\newcommand{\highlight}[1]{\textcolor{red}{\textbf{#1}}}
\newcommand{\code}[1]{\texttt{\small #1}}
\newcommand{\concept}[1]{\textcolor{darkblue}{\textbf{#1}}}

% Footer customization
\setbeamertemplate{footline}[frame number]
\setbeamertemplate{navigation symbols}{}

% Title page customization
\setbeamertemplate{title page}{
    \vbox{}
    \vfill
    \begingroup
        \centering
        \begin{beamercolorbox}[sep=8pt,center,colsep=-4bp,rounded=true,shadow=true]{title}
            \usebeamerfont{title}\inserttitle\par%
            \ifx\insertsubtitle\@empty%
            \else%
                \vskip0.25em%
                {\usebeamerfont{subtitle}\usebeamercolor[fg]{subtitle}\insertsubtitle\par}%
            \fi%
        \end{beamercolorbox}%
        \vskip1em\par
        \begin{beamercolorbox}[sep=8pt,center,colsep=-4bp,rounded=true,shadow=true]{author}
            \usebeamerfont{author}\insertauthor
        \end{beamercolorbox}
        \begin{beamercolorbox}[sep=8pt,center,colsep=-4bp,rounded=true,shadow=true]{institute}
            \usebeamerfont{institute}\insertinstitute
        \end{beamercolorbox}
        \begin{beamercolorbox}[sep=8pt,center,colsep=-4bp,rounded=true,shadow=true]{date}
            \usebeamerfont{date}\insertdate
        \end{beamercolorbox}\vskip0.5em
    \endgroup
    \vfill
}

\begin{document}

% Title slide
\begin{frame}
    \titlepage
\end{frame}

% Table of contents
\begin{frame}
    \frametitle{Course Overview}
    \tableofcontents
\end{frame}

% Introduction slide
\begin{frame}
    \frametitle{Welcome to System Design \& Computer Architecture}
    \begin{itemize}
        \item \textbf{Course Objectives:}
        \begin{itemize}
            \item Understand fundamental computer architecture concepts
            \item Learn system design principles and patterns
            \item Explore modern computing technologies (RISC, ARM, x86)
            \item Apply knowledge to real-world system design problems
        \end{itemize}
        \item \textbf{What You'll Learn:}
        \begin{itemize}
            \item CPU and memory architecture
            \item Instruction set architectures (RISC vs CISC)
            \item System performance optimization
            \item Scalable system design patterns
        \end{itemize}
        \item \textbf{Prerequisites:} Basic programming knowledge, digital logic fundamentals
    \end{itemize}
\end{frame}

% Chapter 1: Computer Organization and Architecture
\section{Chapter 1: The Building Blocks of a Computer - Understanding CPU and Memory Architecture}
\label{sec:chapter_1_coa}
\begin{frame} 
    \frametitle{A Computer's Anatomy}

    \begin{itemize}
        \item \textbf{Objective:} Understand the fundamental components of a computer system, focusing on CPU and memory architecture.
        \item \textbf{Topics Covered:}
        \begin{itemize}
            \item The Von Neumann Architecture
            \item CPU Components
            \item Memory Hierarchy (Cache, RAM, Permanent Storage)
        \end{itemize}
    \end{itemize}
\end{frame}

\begin{frame} 
    \frametitle{Why It Matters}

    \begin{itemize}
        \item \textbf{Performance:} Knowing how CPU and memory interact helps in optimizing software performance.
        \item \textbf{System Design:} Essential for designing efficient systems and applications.
        \item \textbf{Troubleshooting:} Understanding hardware can aid in diagnosing performance bottlenecks.
    \end{itemize}
\end{frame}



\begin{frame} 
    \frametitle{Computer Architecture Overview}
    \begin{itemize}
        \item \textbf{Computer Architecture:} The conceptual design and fundamental operational structure of a computer system.
        \item \textbf{Key Components:}
        \begin{itemize}
            \item \textbf{CPU (Central Processing Unit):} Executes instructions and processes data.
            \item \textbf{Memory:} Stores data and instructions temporarily (RAM) or permanently (SSD/HDD).
            \item \textbf{I/O Devices:} Facilitate interaction with the external environment (keyboard, mouse, display).
        \end{itemize}
        \item \textbf{Data Flow:} The CPU fetches instructions from memory, processes them, and may read/write data to/from memory or I/O devices.
    \end{itemize}
\end{frame}

\begin{frame}
\frametitle{Types of architectures}
\begin{itemize}
    \item \textbf{Von Neumann Architecture:} Single memory space for instructions and data.
    \item \textbf{Harvard Architecture:} Separate memory spaces for instructions and data.
    \item \textbf{ARM Architecture:} A RISC-based design with a load/store model, unified memory, conditional execution, and deep pipelines; widely used in mobile and embedded systems for its power efficiency.
    \item \textbf{Comparison:} Von Neumann is simpler and more common, Harvard can be faster for certain applications, while ARM combines RISC efficiency with a flexible, unified memory system.
\end{itemize}   
\end{frame}

\begin{frame}
\frametitle{Instruction Set Architectures: RISC vs CISC}
\begin{itemize}
    \item \textbf{RISC (Reduced Instruction Set Computer):}
    \begin{itemize}
        \item Simple, uniform instructions (typically 32-bit)
        \item Load/Store architecture - only load/store access memory
        \item One instruction per clock cycle (ideally)
        \item More registers, simpler hardware
    \end{itemize}
    \item \textbf{CISC (Complex Instruction Set Computer):}
    \begin{itemize}
        \item Complex, variable-length instructions
        \item Instructions can directly access memory
        \item Multiple clock cycles per instruction
        \item Fewer registers, more complex hardware
    \end{itemize}
    \item \textbf{Philosophy:} RISC favors simple hardware + smart compilers, CISC favors complex hardware + simple compilers
\end{itemize}
\end{frame}

\begin{frame}
\frametitle{RISC vs CISC: Design Trade-offs}
\begin{columns}[T]
\begin{column}{0.48\textwidth}
\textbf{RISC Advantages:}
\begin{itemize}
    \item Simpler processor design
    \item Lower power consumption
    \item Better pipelining performance
    \item Easier to optimize
    \item Higher clock speeds possible
\end{itemize}
\textbf{RISC Disadvantages:}
\begin{itemize}
    \item More instructions needed
    \item Larger code size
    \item Complex compiler required
\end{itemize}
\end{column}
\begin{column}{0.48\textwidth}
\textbf{CISC Advantages:}
\begin{itemize}
    \item Fewer instructions needed
    \item Smaller code size
    \item Rich instruction set
    \item Backward compatibility
\end{itemize}
\textbf{CISC Disadvantages:}
\begin{itemize}
    \item Complex processor design
    \item Higher power consumption
    \item Difficult to pipeline
    \item Slower clock speeds
\end{itemize}
\end{column}
\end{columns}
\end{frame}

\begin{frame}
\frametitle{ARM Architecture: The RISC Champion}
\begin{itemize}
    \item \textbf{ARM (Advanced RISC Machine):} Dominant RISC architecture
    \begin{itemize}
        \item Founded by Acorn Computers (1985), now ARM Holdings
        \item License-based business model - designs sold to manufacturers
        \item Powers 95\% of smartphones and tablets worldwide
    \end{itemize}
    \item \textbf{Key ARM Characteristics:}
    \begin{itemize}
        \item \textbf{Load/Store Architecture:} Only load/store instructions access memory
        \item \textbf{Fixed 32-bit Instructions:} Uniform instruction length (ARM64: 64-bit)
        \item \textbf{Conditional Execution:} Most instructions can be conditionally executed
        \item \textbf{Low Power Design:} Optimized for battery-powered devices
    \end{itemize}
    \item \textbf{ARM Processor Families:}
    \begin{itemize}
        \item \textbf{Cortex-A:} Application processors (smartphones, tablets)
        \item \textbf{Cortex-R:} Real-time processors (automotive, industrial)
        \item \textbf{Cortex-M:} Microcontrollers (IoT, embedded systems)
    \end{itemize}
\end{itemize}
\end{frame}

\begin{frame}[fragile]
\frametitle{ARM Instruction Set Example}
\begin{itemize}
    \item \textbf{ARM Assembly Examples:}
\end{itemize}
\begin{verbatim}
// Load/Store Operations
LDR R1, [R2]        // Load word from memory[R2] to R1
STR R1, [R2, #4]    // Store R1 to memory[R2 + 4]

// Arithmetic Operations  
ADD R1, R2, R3      // R1 = R2 + R3
SUB R1, R2, #5      // R1 = R2 - 5 (immediate value)

// Conditional Execution
ADDEQ R1, R2, R3    // Add only if equal flag set
MOVNE R1, #0        // Move 0 to R1 if not equal

// Branch Instructions
B label             // Unconditional branch
BEQ label           // Branch if equal
BL function         // Branch with link (function call)
\end{verbatim}
\begin{itemize}
    \item \textbf{Note:} All instructions are 32-bit, enabling efficient pipelining
\end{itemize}
\end{frame}

\begin{frame}
\frametitle{ARM's Modern Success: Apple Silicon}
\begin{itemize}
    \item \textbf{Apple's ARM Transition:}
    \begin{itemize}
        \item \textbf{M1 Chip (2020):} First ARM-based Mac processor
        \item \textbf{M1 Pro/Max (2021):} High-performance variants
        \item \textbf{M2 Series (2022+):} Next generation ARM processors
    \end{itemize}
    \item \textbf{ARM Advantages in Apple Silicon:}
    \begin{itemize}
        \item \textbf{Power Efficiency:} Exceptional battery life in MacBooks
        \item \textbf{Unified Memory:} CPU and GPU share same memory pool
        \item \textbf{Custom Silicon:} Apple designs custom ARM cores
        \item \textbf{Performance:} Competitive with Intel/AMD x86 processors
    \end{itemize}
    \item \textbf{Market Impact:}
    \begin{itemize}
        \item Proved ARM can compete in laptop/desktop market
        \item Microsoft developing ARM-based Windows
        \item Amazon's Graviton ARM servers gaining adoption
    \end{itemize}
\end{itemize}
\end{frame}

\begin{frame}
\frametitle{x86 Architecture: The CISC Powerhouse}
\begin{itemize}
    \item \textbf{x86 History:}
    \begin{itemize}
        \item \textbf{Intel 8086 (1978):} Original 16-bit processor
        \item \textbf{80386 (1985):} First 32-bit x86 processor
        \item \textbf{x86-64/AMD64 (2003):} 64-bit extension by AMD
        \item Dominates desktop, laptop, and server markets
    \end{itemize}
    \item \textbf{x86 CISC Characteristics:}
    \begin{itemize}
        \item \textbf{Variable Instruction Length:} 1 to 15 bytes per instruction
        \item \textbf{Complex Instructions:} Single instruction can do multiple operations
        \item \textbf{Memory-to-Memory Operations:} Direct memory manipulation
        \item \textbf{Rich Addressing Modes:} Multiple ways to specify operands
    \end{itemize}
    \item \textbf{Modern x86 Complexity:}
    \begin{itemize}
        \item Hundreds of instructions in instruction set
        \item Backward compatibility maintained since 8086
        \item Internal RISC-like execution (micro-ops)
    \end{itemize}
\end{itemize}
\end{frame}

\begin{frame}[fragile]
\frametitle{x86 Instruction Set Example}
\begin{itemize}
    \item \textbf{x86 Assembly Examples:}
\end{itemize}
\begin{verbatim}
// Complex Memory Operations
ADD [EBX], EAX      // Add EAX to memory[EBX], store in memory
MOV EAX, [EBX+4]    // Load from memory[EBX+4] to EAX

// Variable Length Instructions
MOV AL, 5           // 2 bytes: Move immediate to 8-bit register
MOV EAX, 0x12345678 // 5 bytes: Move 32-bit immediate to register

// Complex Addressing Modes
MOV EAX, [EBX + ECX*2 + 8]  // EAX = memory[EBX + ECX*2 + 8]

// String Operations
REP MOVSB          // Repeat move string bytes (hardware loop)

// Stack Operations
PUSH EAX           // Push EAX onto stack
CALL function      // Call function (push return address + jump)
\end{verbatim}
\begin{itemize}
    \item \textbf{Note:} Instructions vary from 1-15 bytes, complex decoding required
\end{itemize}
\end{frame}

\begin{frame}
\frametitle{Modern x86: CISC Outside, RISC Inside}
\begin{itemize}
    \item \textbf{Micro-Operation Translation:}
    \begin{itemize}
        \item Complex x86 instructions decoded into simple micro-ops
        \item Internal execution core is RISC-like
        \item Best of both worlds: CISC compatibility + RISC performance
    \end{itemize}
    \item \textbf{Example Translation:}
    \begin{itemize}
        \item \texttt{ADD [EBX], EAX} becomes:
        \item \texttt{LOAD temp, [EBX]} (micro-op 1)
        \item \texttt{ADD temp, EAX} (micro-op 2)  
        \item \texttt{STORE temp, [EBX]} (micro-op 3)
    \end{itemize}
    \item \textbf{Performance Techniques:}
    \begin{itemize}
        \item \textbf{Out-of-Order Execution:} Execute micro-ops as dependencies allow
        \item \textbf{Superscalar:} Multiple execution units run in parallel
        \item \textbf{Branch Prediction:} Predict which way branches will go
        \item \textbf{Speculative Execution:} Execute ahead speculatively
    \end{itemize}
\end{itemize}
\end{frame}

\begin{frame}
\frametitle{ARM vs x86: Performance and Power Comparison}
\begin{columns}[T]
\begin{column}{0.48\textwidth}
\textbf{ARM Strengths:}
\begin{itemize}
    \item \textbf{Power Efficiency:} 3-5x better performance per watt
    \item \textbf{Heat Generation:} Runs cooler, enables fanless designs
    \item \textbf{Battery Life:} Exceptional in mobile devices
    \item \textbf{Custom Silicon:} Licensees can customize designs
    \item \textbf{Cost:} Lower licensing and manufacturing costs
\end{itemize}
\end{column}
\begin{column}{0.48\textwidth}
\textbf{x86 Strengths:}
\begin{itemize}
    \item \textbf{Raw Performance:} Higher peak performance in many workloads
    \item \textbf{Software Ecosystem:} Decades of optimized software
    \item \textbf{Enterprise Features:} Advanced virtualization, security
    \item \textbf{Backward Compatibility:} Runs legacy software unchanged
    \item \textbf{Manufacturing:} Advanced process nodes (Intel, TSMC)
\end{itemize}
\end{column}
\end{columns}
\begin{itemize}
    \item \textbf{Current Trend:} ARM gaining ground in servers and laptops, x86 still dominant in desktop/enterprise
    \item \textbf{Future:} Likely convergence with both architectures borrowing from each other
\end{itemize}
\end{frame}

\begin{frame}
\frametitle{Real-World Applications: Choosing the Right Architecture}
\begin{itemize}
    \item \textbf{ARM Dominates:}
    \begin{itemize}
        \item \textbf{Mobile Devices:} Smartphones, tablets (95\% market share)
        \item \textbf{IoT/Embedded:} Sensors, smart devices, automotive
        \item \textbf{Apple Ecosystem:} M1/M2 MacBooks, iPhones, iPads
        \item \textbf{Cloud Computing:} Amazon Graviton, custom server chips
    \end{itemize}
    \item \textbf{x86 Dominates:}
    \begin{itemize}
        \item \textbf{Desktop/Laptop PCs:} Gaming, productivity, development
        \item \textbf{Enterprise Servers:} Data centers, high-performance computing
        \item \textbf{Legacy Systems:} Existing infrastructure and software
        \item \textbf{High-End Gaming:} Maximum performance requirements
    \end{itemize}
    \item \textbf{Decision Factors:}
    \begin{itemize}
        \item Power efficiency vs raw performance
        \item Software compatibility requirements
        \item Cost constraints and development timeline
        \item Target market and use case
    \end{itemize}
\end{itemize}
\end{frame}



\begin{frame}  
    \frametitle{The Von Neumann Architecture}
\begin{itemize}
    \item \textbf{The Von Neumann Architecture:} The core model of a modern computer.
    \begin{itemize}
        \item Central Processing Unit (CPU): The "brain."
        \item Main Memory (RAM): The workspace.
        \item Input/Output (I/O) Systems.
    \end{itemize}
    \item \textbf{A Deeper Look at the CPU:}
    \begin{itemize}
        \item Control Unit (CU), Arithmetic Logic Unit (ALU), Registers.
    \end{itemize}
    \item \textbf{The Memory Hierarchy:} A pyramid of speed, cost, and size.
    \begin{itemize}
        \item \textbf{L1/L2/L3 Cache:} Ultra-fast memory on the CPU.
        \item \textbf{RAM (Random Access Memory):} Volatile, fast memory for active programs.
        \item \textbf{Permanent Storage:} Non-volatile, slower storage (SSDs, HDDs).
    \end{itemize}
\end{itemize}
\end{frame}  



% Future chapters can be added here
% \input{chapter_2_memory.tex}
% \input{chapter_3_io.tex}
% \input{chapter_4_parallel.tex}
% \input{chapter_5_performance.tex}
% \input{chapter_6_system_design.tex}

% Summary and conclusion
\section{Course Summary}

\begin{frame}
    \frametitle{Key Takeaways}
    \begin{itemize}
        \item \textbf{Architecture Matters:}
        \begin{itemize}
            \item RISC vs CISC trade-offs shape modern computing
            \item ARM's power efficiency revolutionizing mobile and laptop markets
            \item x86's complexity enables high performance in servers and desktops
        \end{itemize}
        \item \textbf{System Design Principles:}
        \begin{itemize}
            \item Understand your hardware constraints and capabilities
            \item Choose the right architecture for your use case
            \item Balance performance, power, and cost requirements
        \end{itemize}
        \item \textbf{Future Trends:}
        \begin{itemize}
            \item ARM expanding into server and desktop markets
            \item Heterogeneous computing (CPU + GPU + specialized processors)
            \item Quantum and neuromorphic computing on the horizon
        \end{itemize}
    \end{itemize}
\end{frame}

\begin{frame}
    \frametitle{Next Steps}
    \begin{itemize}
        \item \textbf{Hands-On Practice:}
        \begin{itemize}
            \item Experiment with ARM and x86 assembly language
            \item Profile applications to understand performance bottlenecks
            \item Design systems with different architectural constraints
        \end{itemize}
        \item \textbf{Further Learning:}
        \begin{itemize}
            \item Advanced computer architecture courses
            \item System design interview preparation
            \item Open-source hardware projects (RISC-V)
            \item High-performance computing and parallel programming
        \end{itemize}
        \item \textbf{Career Applications:}
        \begin{itemize}
            \item System architecture roles
            \item Performance engineering
            \item Embedded systems development
            \item Cloud infrastructure design
        \end{itemize}
    \end{itemize}
\end{frame}

% References and resources
\begin{frame}
    \frametitle{References and Resources}
    \begin{itemize}
        \item \textbf{Textbooks:}
        \begin{itemize}
            \item Computer Organization and Design - Patterson \& Hennessy
            \item Computer Architecture: A Quantitative Approach - Hennessy \& Patterson
            \item ARM System Developer's Guide - Sloss, Symes \& Wright
        \end{itemize}
        \item \textbf{Online Resources:}
        \begin{itemize}
            \item ARM Developer Documentation: \url{https://developer.arm.com}
            \item Intel x86 Architecture Manuals: \url{https://intel.com/sdm}
            \item RISC-V Foundation: \url{https://riscv.org}
        \end{itemize}
        \item \textbf{Tools and Simulators:}
        \begin{itemize}
            \item QEMU for architecture emulation
            \item ARM Development Studio
            \item Intel VTune Profiler
            \item Online assembly simulators and debuggers
        \end{itemize}
    \end{itemize}
\end{frame}

% Thank you slide
\begin{frame}
    \frametitle{Thank You!}
    \begin{center}
        \Huge Questions?
        
        \vspace{1cm}
        
        \large
        Contact Information:\\
        Email: your.email@institution.edu\\
        Office Hours: By appointment
        
        \vspace{1cm}
        
        \normalsize
        \textit{``The best way to learn computer architecture is to build one.''}\\
        \textit{-- Anonymous Computer Architect}
    \end{center}
\end{frame}

\end{document}
