\subsubsection{Visualizations \& Diagrams}
\textbf{Von Neumann Architecture:}
\begin{minted}{mermaid}
graph TD
    subgraph Computer
        CPU -- "fetches/stores" --> Memory;
        CPU -- "reads/writes" --> IO_Devices;
        IO_Devices -- "data" --> Memory;
    end
    CPU <--> CU & ALU & Registers;
    subgraph IO_Devices
        direction LR
        Input_Device --> CPU;
        CPU --> Output_Device;
    end
\end{minted}

\textbf{Memory Hierarchy:}
\begin{minted}{mermaid}
graph TD
    A[CPU Registers] --> B(L1 Cache);
    B --> C(L2 Cache);
    C --> D(L3 Cache);
    D --> E(Main Memory - RAM);
    E --> F(Permanent Storage - SSD/HDD);
\end{minted}

\subsubsection{Real-World Application}
\textbf{Launching an Application:} When you double-click an icon for a program (e.g., a web browser), the OS finds the program on your \textbf{Permanent Storage (SSD)} and loads it into \textbf{Main Memory (RAM)}. The \textbf{CPU} then fetches instructions and data from RAM into its \textbf{Caches} to execute the program, drawing the user interface on your screen via an \textbf{Output Device}.

\subsubsection{References \& Further Reading}
\begin{itemize}
    \item \textbf{Video:} Crash Course Computer Science - The CPU: \url{https://www.youtube.com/watch?v=cNN_tTXABUA}
    \item \textbf{Article:} "How Computers Work: The CPU and Memory" by Code.org: \url{https://code.org/files/curriculum/course4/10_HowComputersWork.pdf}
\end{itemize}

\subsection{Chapter 2: Processing in Parallel - Advanced CPU Architectures}
\subsubsection{Objective}
Explore how modern processors handle multiple tasks and data streams simultaneously.

\subsubsection{Topics}
\begin{itemize}
    \item \textbf{Introduction to Flynn's Taxonomy:} A classification of computer architectures.
    \item \textbf{SIMD (Single Instruction, Multiple Data):} One instruction applied to many data points.
    \item \textbf{MIMD (Multiple Instruction, Multiple Data):} Multiple instructions on multiple data streams.
\end{itemize}

\subsubsection{Visualizations \& Diagrams}
\begin{minted}{mermaid}
graph TD
    subgraph SIMD
        direction LR
        Instruction_SIMD --> Data1;
        Instruction_SIMD --> Data2;
        Instruction_SIMD --> Data3;
    end
    subgraph MIMD
        direction LR
        Instruction1_MIMD --> DataA;
        Instruction2_MIMD --> DataB;
        Instruction3_MIMD --> DataC;
    end
\end{minted}

\subsubsection{Real-World Application}
\begin{itemize}
    \item \textbf{SIMD:} Applying a brightness filter in a photo editor. The \textit{same instruction} ("increase brightness by 10\%") is applied to \textit{every pixel} in the image (multiple data) at once, often by the GPU.
    \item \textbf{MIMD:} A modern web server handling requests from three different users simultaneously. One core (or thread) processes User A's request for a profile page, a second core processes User B's file upload, and a third processes User C's database query. Each is a different instruction on different data.
\end{itemize}

\subsubsection{References \& Further Reading}
\begin{itemize}
    \item \textbf{Video:} Flynn's Taxonomy Explained: \url{https://www.youtube.com/watch?v=1OLAl9M7pmE}
    \item \textbf{Image:} A visual of SIMD vs MIMD: \url{https://www.researchgate.net/profile/Jose-Gracia/publication/273764335/figure/fig1/AS:669459278868483@1536622952381/Illustration-of-the-SIMD-and-MIMD-parallel-computing-paradigms.png}
\end{itemize}

