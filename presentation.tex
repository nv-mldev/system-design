\documentclass{beamer}
\usepackage{graphicx}
\usepackage{hyperref}
\usepackage{verbatim}

% Theme choice
\usetheme{Madrid}
\usecolortheme{default}

% Title and Author Information
\title{An Introduction to System Integration}
\author{Gemini}
\institute{Software Team}
\date{\today}

\begin{document}

% Title Frame
\begin{frame}
  \titlepage
\end{frame}

% Table of Contents
\begin{frame}
  \frametitle{Outline}
  \tableofcontents
\end{frame}

% --- Chapter 1 ---
\section{Computer Architecture Fundamentals}

\begin{frame}
  \frametitle{Chapter 1: The Heart of the Machine}
  \begin{itemize}
    \item \textbf{Objective:} Understand the primary components of a single computer and how they interact.
  \end{itemize}
\end{frame}

\begin{frame}[fragile]
  \frametitle{Topics: Core Architecture}
  \begin{itemize}
    \item \textbf{The Von Neumann Architecture:}
    \begin{itemize}
        \item Central Processing Unit (CPU)
        \item Main Memory (RAM)
        \item Input/Output (I/O) Systems
    \end{itemize}
    \item \textbf{A Deeper Look at the CPU:}
    \begin{itemize}
        \item Control Unit (CU), Arithmetic Logic Unit (ALU), Registers
    \end{itemize}
    \item \textbf{The Memory Hierarchy:}
    \begin{itemize}
        \item L1/L2/L3 Cache
        \item RAM (Random Access Memory)
        \item Permanent Storage (SSDs, HDDs)
    \end{itemize}
  \end{itemize}
\end{frame}

\begin{frame}[fragile]
  \frametitle{Diagram: Von Neumann Architecture}
  \begin{verbatim}
  graph TD
      subgraph Computer
          CPU -- "fetches/stores" --> Memory;
          CPU -- "reads/writes" --> IO_Devices;
          IO_Devices -- "data" --> Memory;
      end
      CPU <--> CU & ALU & Registers;
      subgraph IO_Devices
          direction LR
          Input_Device --> CPU;
          CPU --> Output_Device;
      end
  \end{verbatim}
\end{frame}

\begin{frame}[fragile]
  \frametitle{Diagram: Memory Hierarchy}
  \begin{verbatim}
  graph TD
      A[CPU Registers] --> B(L1 Cache);
      B --> C(L2 Cache);
      C --> D(L3 Cache);
      D --> E(Main Memory - RAM);
      E --> F(Permanent Storage - SSD/HDD);
  \end{verbatim}
\end{frame}

\begin{frame}
  \frametitle{Real-World Application: Launching an App}
  \begin{itemize}
    \item When you double-click an icon:
    \item The OS finds the program on your \textbf{Permanent Storage (SSD)}.
    \item It loads the program into \textbf{Main Memory (RAM)}.
    \item The \textbf{CPU} fetches instructions from RAM into its \textbf{Caches} to execute the program.
  \end{itemize}
\end{frame}

% --- Chapter 2 ---
\section{Advanced CPU Architectures}

\begin{frame}
  \frametitle{Chapter 2: Processing in Parallel}
  \begin{itemize}
    \item \textbf{Objective:} Explore how modern processors handle multiple tasks and data streams simultaneously.
  \end{itemize}
\end{frame}

\begin{frame}[fragile]
  \frametitle{Topics: Flynn's Taxonomy}
  \begin{itemize}
    \item \textbf{SIMD (Single Instruction, Multiple Data):}
    \begin{itemize}
        \item One instruction is applied to many different data points at once.
        \item \textbf{Analogy:} A drill sergeant telling a whole platoon to "turn left".
    \end{itemize}
    \item \textbf{MIMD (Multiple Instruction, Multiple Data):}
    \begin{itemize}
        \item Multiple processors execute different instructions on different data streams.
        \item \textbf{Analogy:} A workshop with multiple craftspeople working on different projects.
    \end{itemize}
  \end{itemize}
\end{frame}

\begin{frame}[fragile]
  \frametitle{Diagram: SIMD vs MIMD}
  \begin{verbatim}
  graph TD
      subgraph SIMD
          direction LR
          Instruction_SIMD --> Data1;
          Instruction_SIMD --> Data2;
          Instruction_SIMD --> Data3;
      end
      subgraph MIMD
          direction LR
          Instruction1_MIMD --> DataA;
          Instruction2_MIMD --> DataB;
          Instruction3_MIMD --> DataC;
      end
  \end{verbatim}
\end{frame}

\begin{frame}
  \frametitle{Real-World Application: SIMD vs MIMD}
  \begin{itemize}
    \item \textbf{SIMD:} Applying a brightness filter in a photo editor. The same instruction ("increase brightness") is applied to every pixel at once.
    \item \textbf{MIMD:} A web server handling multiple user requests simultaneously. Each core processes a different request.
  \end{itemize}
\end{frame}

% --- Chapter 3 ---
\section{Network Architectures and Layers}

\begin{frame}
  \frametitle{Chapter 3: Connecting the Dots}
  \begin{itemize}
    \item \textbf{Objective:} Understand how different computer systems communicate with each other over a network.
  \end{itemize}
\end{frame}

\begin{frame}[fragile]
  \frametitle{Topics: The TCP/IP Model}
  A 5-Layer View for simplifying the complexity of networking:
  \begin{itemize}
    \item Layer 5: Application (HTTP, DNS)
    \item Layer 4: Transport (TCP, UDP)
    \item Layer 3: Network (IP)
    \item Layer 2: Data Link (Ethernet, Wi-Fi)
    \item Layer 1: Physical (Cables, Radio Waves)
  \end{itemize}
\end{frame}

\begin{frame}[fragile]
  \frametitle{Diagram: TCP/IP Model}
  \begin{verbatim}
  graph TD
      subgraph Your Computer
          A[Application] --> B(Transport);
          B --> C(Network);
          C --> D(Data Link);
          D --> E(Physical);
      end
      subgraph Web Server
          F[Application] --> G(Transport);
          G --> H(Network);
          H --> I(Data Link);
          I --> J(Physical);
      end
      E -- The Internet --> J;
  \end{verbatim}
\end{frame}

% --- Chapter 4 ---
\section{A Practical Example}

\begin{frame}
  \frametitle{Chapter 4: Tying It All Together}
  \begin{itemize}
    \item \textbf{Objective:} Trace a single, common action from start to finish to see how all the concepts interact.
    \item \textbf{Scenario:} "Loading google.com in a web browser."
  \end{itemize}
\end{frame}

\begin{frame}[fragile]
  \frametitle{Diagram: Loading google.com}
  \begin{verbatim}
   sequenceDiagram
      participant User
      participant Browser
      participant OS
      participant DNS_Server
      participant Google_Server

      User->>Browser: Enters "google.com"
      Browser->>OS: Need IP for "google.com"
      OS->>DNS_Server: Where is "google.com"?
      DNS_Server-->>OS: IP is 142.250.190.78
      OS-->>Browser: Here is the IP
      Browser->>Google_Server: HTTP GET request
      Google_Server-->>Browser: HTTP 200 OK
      Browser->>User: Renders the webpage
  \end{verbatim}
\end{frame}

% --- Chapter 5 ---
\section{Network Communication Patterns}

\begin{frame}
  \frametitle{Chapter 5: Network Communication}
  \begin{itemize}
    \item \textbf{Objective:} Explore common methods and patterns for communication between systems.
  \end{itemize}
\end{frame}

\begin{frame}[fragile]
  \frametitle{Topics: Communication Patterns}
  \begin{itemize}
    \item \textbf{HTTP Communication:} Request/Response model.
    \item \textbf{Sockets:} Low-level, bidirectional communication.
    \item \textbf{Web Servers:} Role of servers like Nginx.
    \item \textbf{Message Queues:} Decoupling systems.
    \item \textbf{Publish-Subscribe Pattern:} Scalable messaging.
  \end{itemize}
\end{frame}

\begin{frame}[fragile]
  \frametitle{Diagram: Message Queue vs Pub/Sub}
  \begin{verbatim}
  graph TD
      subgraph Message Queue
          Producer --> Queue((Queue));
          Queue --> Consumer;
      end
      subgraph Publish-Subscribe
          Publisher --> Broker((Topic));
          Broker --> Subscriber1;
          Broker --> Subscriber2;
      end
  \end{verbatim}
\end{frame}

% --- Chapter 6 ---
\section{Concurrency and Parallelism}

\begin{frame}
  \frametitle{Chapter 6: Concurrency & Parallelism}
  \begin{itemize}
    \item \textbf{Objective:} Understand different models for executing multiple tasks at the same time.
  \end{itemize}
\end{frame}

\begin{frame}[fragile]
  \frametitle{Topics: Concurrency Models}
  \begin{itemize}
    \item \textbf{Concurrent Processing (Threads):} Independent execution paths in one process. Feels simultaneous.
    \item \textbf{Parallel Processing:} Truly simultaneous execution on multiple cores.
    \item \textbf{Asynchronous Processing (Async):} Non-blocking operations.
  \end{itemize}
\end{frame}

\begin{frame}[fragile]
  \frametitle{Diagram: Concurrency vs Parallelism}
  \begin{verbatim}
  graph TD
      subgraph Concurrency (1 Core)
          direction LR
          Core1 -- Task A --> Switch;
          Switch -- Task B --> Switch;
          Switch -- Task A --> ...;
      end
      subgraph Parallelism (2 Cores)
          direction LR
          CoreA -- Task A --> DoneA;
          CoreB -- Task B --> DoneB;
      end
  \end{verbatim}
\end{frame}

% --- Chapter 7 ---
\section{Operating System Fundamentals}

\begin{frame}
  \frametitle{Chapter 7: OS Fundamentals}
  \begin{itemize}
    \item \textbf{Objective:} Gain a foundational understanding of the role of the Operating System.
  \end{itemize}
\end{frame}

\begin{frame}[fragile]
  \frametitle{Topics: OS Core Functions}
  \begin{itemize}
    \item Process Management
    \item Memory Management
    \item File Systems
    \item I/O Handling
    \item Windows vs. Linux
    \item Hyper-Threading
  \end{itemize}
\end{frame}

\begin{frame}[fragile]
  \frametitle{Diagram: OS Kernel}
  \begin{verbatim}
  graph TD
      subgraph OS Kernel
          Scheduler --> P1(Process 1);
          Scheduler --> P2(Process 2);
          MemoryManager --> P1_Mem;
          MemoryManager --> P2_Mem;
      end
      subgraph Hardware
          CPU & RAM & Disk;
      end
      OS_Kernel -- "manages" --> Hardware;
  \end{verbatim}
\end{frame}

% --- Chapter 8 ---
\section{Virtualization and Isolation}

\begin{frame}
  \frametitle{Chapter 8: Virtualization}
  \begin{itemize}
    \item \textbf{Objective:} Understand how we create virtual environments to run software.
  \end{itemize}
\end{frame}

\begin{frame}[fragile]
  \frametitle{Topics: Virtualization Types}
  \begin{itemize}
    \item \textbf{Virtual Machines (VMs):} Emulating an entire computer system (hardware + OS).
    \item \textbf{Containers:} OS-level virtualization, packaging an application and its dependencies (e.g., Docker).
    \item Key differences: Overhead, startup time, density.
  \end{itemize}
\end{frame}

\begin{frame}[fragile]
  \frametitle{Diagram: VMs vs Containers}
  \begin{figure}
    \includegraphics[width=\textwidth]{mermaid-diagrams/09_vms_vs_containers.png}
    \caption{VMs vs Containers}
  \end{figure}
\end{frame}

% --- Final Frame ---
\begin{frame}
  \Huge{\centerline{Thank You}}
\end{frame}

\end{document}
ontainerEngine --> App_D;
          end
      end
  \end{verbatim}
\end{frame}

% --- Final Frame ---
\begin{frame}
  \Huge{\centerline{Thank You}}
\end{frame}

\end{document}
ent}
uge{\centerline{Thank You}}
\end{frame}

\end{document}
e ---
\begin{frame}
  \Huge{\centerline{Thank You}}
\end{frame}

\end{document}
